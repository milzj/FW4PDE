\documentclass{scrartcl}


\usepackage{scrhack}
\usepackage[utf8]{inputenc}
\usepackage{mathtools}
\usepackage{mathrsfs}
\usepackage{amsmath}
\usepackage{amsfonts}
\usepackage{amssymb}
\usepackage{amsthm}
\usepackage{enumitem}

\usepackage{hyperref}

\title{Adaptive step sizes for conditional gradient methods}


\author{Johannes Milz}


\date{January 22, 2023}

\begin{document}
	
	\maketitle
	
	We establish convergence rates for step size adaptive conditional
	gradient methods as applied to nonconvex, smooth optimization.
	These step size rules are implemented in 
	\url{https://github.com/milzj/FW4PDE}.
	
	We use the same notation and assumptions 
	as those used in \cite[sect.\ 7.3]{Lan2020}. 
	Moreover, we define $y_k(\alpha) = (1-\alpha)y_{k-1} + \alpha x_k$.
	Throughout the text, we use the fact that $\mathrm{gap}(y_{k-1}) \geq 0$.
	
	\paragraph{Adaptive ``estimation'' of derivative's Lipschitz constant}
	We compute the step size$\alpha_k \in [0,1]$ according to the following
	scheme inspired by the line search-type scheme used in \cite{Nesterov2006}
	(see also \cite{Pedregosa2018,Pedregosa2020}).
	Let $f'$ be $\|\cdot\|$-Lipschitz continuous with Lipschitz
	constant $L$. Moreover, let $L_{-1} \leq L$.
	\begin{enumerate}
		\item 
		Compute $\alpha_k^*$ via minimizing
		the model function
		\begin{align*}
			\phi_k(\alpha) = 
			f(y_{k-1}) - \alpha \mathrm{gap}(y_{k-1})
			+ \alpha^2(L_k/2) \|x_k -y_{k-1}\|^2
		\end{align*}
		over $[0,1]$.
		
		\item If 
		\begin{align*}
			f(y_k(\alpha_k^*)) \leq  \phi_k(\alpha_k^*),
		\end{align*}
		then define $y_k = y_k(\alpha_k^*)$
		and $L_{k+1}  = \max\{L_{-1},L_k/2\}$.
		Otherwise, define $L_{k} = 2L_k$ and go to the first step.
	\end{enumerate}
	For each iteration $k$, we have $L_{-1} \leq L_k \leq 2L$.
	
	If $\|x_k -y_{k-1}\| = 0$, then $\mathrm{gap}(y_{k-1}) = 0$.
	In this case $\alpha_k^*= 0$. Otherwise`
	\begin{align*}
		\alpha_k^* = \min\{1, \frac{\mathrm{gap}(y_{k-1})}{L_k\|x_k -y_{k-1}\|^2}\} 
	\end{align*}
	The choice of $L_k$ ensures that
	$0 < 1/(2L) \leq 1/L_k \leq 1/L_{-1}$.
	So the setting
	presented here differs from that considered in \cite[p.\ 24]{Levitin1966}.


	Otherwise, 
	\begin{align*}
		f(y_{k}) \leq f(y_{k-1}) 
		- \frac{\mathrm{gap}(y_{k-1})^2}{2L_k \|x_k -y_{k-1}\|^2}.
	\end{align*}
	Since $\|x_k -y_{k-1}\| \leq \bar{D}_{X}$ and $2L \geq L_k$, we obtain
	\begin{align*}
		\frac{\mathrm{gap}(y_{k-1})^2}{4L\bar{D}_{X}^2}
		\leq 
		\frac{\mathrm{gap}(y_{k-1})^2}{2L_k \|x_k -y_{k-1}\|^2}
		\leq f(y_{k-1})  - f(y_k).
	\end{align*}
	Hence
	\begin{align*}
		\sum_{k=1}^{K} \mathrm{gap}(y_{k-1})^2
		\leq 4 L \bar{D}_{X}^2(f(y_0)-f(y_K)).
	\end{align*}
	Hence $\mathrm{gap}(y_{k-1}) \to 0$ as $k \to \infty$ and
	\begin{align*}
		\min_{1\leq k \leq K}\, \mathrm{gap}(y_{k-1}) 
		\leq \sqrt{\frac{4 L \bar{D}_{X}^2(f(y_0)-f^*)}{K}}.
	\end{align*}
	This bound is slightly worse to that established in \cite[p.\ 469]{Lan2020}
	with $\alpha_k$ chosen proportional to $1/\sqrt{k}$.
	However, $\alpha_k^*$ may take values close or equal to one.
	
	\paragraph{quasi-Armijo--Goldstein line search}
	We follow \cite[Def.\ 3.2]{Kunisch2022}. Let $\gamma \in (0,1)$ and 
	$\rho \in (0,1/2]$. We compute $\alpha_k = \gamma^{n_k}$,
	where $n_k \in \{0\} \cup \mathbb{N}$ is the smallest integer with
	\begin{align*}
		f(y_k(\alpha_k)) \leq f(y_{k-1}) - \rho \alpha_k \mathrm{gap}(y_{k-1}).
	\end{align*} 
	Subsequently, we define $y_k = y_k(\alpha_k)$.
	


	Using the $\|\cdot\|$-Lipschitz continuity of $f'$ and the derivations in 
	\cite[p.\ 469]{Lan2020}, we obtain
	\begin{align*}
		f(y_k(\gamma^{-1}\alpha_k)) \leq f(y_{k-1}) - 
		\gamma^{-1}\alpha_k \mathrm{gap}(y_{k-1})
		+ \alpha_k^2 \gamma^{-2} (L/2) \bar{D}_X^2.
	\end{align*}
	Using the Armijo--Goldstein condition, we have
	\begin{align*}
		f(y_k(\gamma^{-1}\alpha_k)) > f(y_{k-1}) - \rho \gamma^{-1}
		\alpha_k \mathrm{gap}(y_{k-1}).
	\end{align*}
	Combining these estimates, we obtain
	\begin{align*}
		f(y_{k-1}) - \rho \gamma^{-1}
		\alpha_k \mathrm{gap}(y_{k-1}) < 
		f(y_{k-1}) - 
		\gamma^{-1}\alpha_k \mathrm{gap}(y_{k-1})
		+ \alpha_k^2 \gamma^{-2} (L/2) \bar{D}_X^2.
	\end{align*}
	Hence
	\begin{align*}
		\mathrm{gap}(y_{k-1}) \cdot (1-\rho) < 
		\alpha_k\gamma^{-1} (L/2) \bar{D}_X^2.
	\end{align*}
	Using the Armijo--Goldstein condition, we have
	\begin{align*}
		f(y_{k-1}) - f(y_k) \geq  \rho \alpha_k \mathrm{gap}(y_{k-1}).
	\end{align*}
	Combining the estimates, we obtain
	\begin{align*}
		f(y_{k-1}) - f(y_k) \geq  
		\frac{2\rho(1-\rho)\gamma}{L\bar{D}_X^2}
		\mathrm{gap}(y_{k-1})^2
	\end{align*}
	Hence
	\begin{align*}
		\min_{1\leq k \leq K}\, \mathrm{gap}(y_{k-1}) 
		\leq 
		\sqrt{\frac{L\bar{D}_X^2(f(y_0)-f^*)}{2\rho(1-\rho)\gamma \cdot K}}.
	\end{align*}
	
	\paragraph{Using a second-order Taylor's expansion
	and minimization rule}
	
	If $f$ is twice differentiable and quadratic,
	and computing the second derivative of $f$
	(in the directions $x_k-y_{k-1}$)
	is computationally feasible, then we use the minimization rule to compute
	the optimal step size $\alpha_k$.

	If $f$ is twice differentiable, we have
	\begin{align*}
		f(y_k(\alpha)) \approx f(y_{k-1}) - \alpha \mathrm{gap}(y_{k-1})
		+ \alpha^2 (1/2) f''(y_{k-1})[x_k-y_{k-1}]^2.
	\end{align*}
	If $\approx$ is $\leq$, then $\alpha_k$ computed using 
	the minimization rule should yield a convergent method.
	Whether $\leq$ nor not can be verified computationally. 
	
	If $\approx$ is $\geq 0$, then $\alpha_k$ computed using 
	the minimization rule may be a poor choice. 
	We may use regularization of $f''$, use an offline choice
	for $\alpha_k$, or use an Armijo step size. 
	The step size computed using minimization rule should be better than 
	a step size computed using Armijo rule.
	

	However, as long as for some $\rho \in (0,1)$, 
	\begin{align*}
		\sum_{k=1}^K \rho \alpha_k \mathrm{gap}(y_{k-1})
		\leq f(y_0) - f(y_K)
	\end{align*}
	we should be fine. This estimate allows for nonmonotone
	function values.
	
	
	Alternative approach
	\begin{enumerate}
		\item Compute $\alpha_k^*$ using minimization rule applied
		to quadratic model.
		\item Check if 
		$f(y_k(\alpha_k^*)) \leq $
		quadratic model evaluated at 
		$\alpha_k^*$.
		If it fails, use  quasi-Armijo--Goldstein step size 
		(starting with largest
		$n_k$ such that $\gamma^{n_k} \leq \alpha_k^*$.
		Choose small $\rho > 0$.)
	\end{enumerate}
	
	
	
	
	\bibliographystyle{myplainurl}
	\bibliography{lit}
	
\end{document}